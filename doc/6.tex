\documentclass[serif, 12pt]{beamer}

\usepackage{graphicx} % Allows including images
\usepackage{booktabs} % Allows the use of \toprule, \midrule and \bottomrule in tables

\usepackage{color}

%\usepackage{algorithm2e}
\usepackage{hyperref}
\usepackage{algorithm,algorithmic}
\usepackage{changepage}
\usepackage{pgfplots}
\usepackage{pgfplotstable}
\usepgfplotslibrary{statistics}

\newcommand*\mat[1]{ \begin{pmatrix} #1 \end{pmatrix}}
\newcommand*\arr[1]{ \begin{bmatrix} #1 \end{bmatrix}}
\newcommand*\V[1]{ \boldsymbol{#1}}

\newcommand*\D{\textcolor{violet}{D}}
\newcommand*\T{\textcolor{blue}{T}}

\setbeamertemplate{navigation symbols}{}%remove navigation symbols

\setbeamerfont{page number in head/foot}{size=\small}
\setbeamertemplate{footline}[frame number]

\title{Approximated PCA}
\subtitle{Iteration 6}

\author{Rodrigo Arias} % Your name
\date{\today} % Date, can be changed to a custom date

\begin{document}

\begin{frame}
	\titlepage
\end{frame}

%------------------------------------------------

\begin{frame}

\frametitle{Reduction of space in Householder}

The Householder algorithm needs 6 internal f.p. variables to perform the 
tridiagonalization.

\vspace{1em}

Also, 3 arguments are reused to store the results:
\begin{itemize}
\item The matrix $A$ with size $n\times n$
\item The diagonal $d$ with size $n$
\item The offdiagonal $o$ with size $n$
\end{itemize}

\vspace{1em}

In total, 9 variables are used.

\end{frame}

%------------------------------------------------

\begin{frame}

\frametitle{Individual precision}

\begin{itemize}
\item The 3 arguments have the bigger impact (multiple elements) as shown by 
previous experiments.
\item Different precisions can be set to individual elements.
\item We can compare results between different configurations.
\end{itemize}

\end{frame}

%------------------------------------------------

\begin{frame}
\frametitle{Experiment H}

A new experiment can be designed, to test individual precisions.
\vspace{1em}

\begin{itemize}

\item Each element is assigned a random precision from $C = \{8, 16, 32, 64\}$, 
with equal probability
\item The other variables are kept at 64 bits.
\item The storage size and the error is plotted.

\end{itemize}

\end{frame}

%------------------------------------------------

\begin{frame}

\centering
\includegraphics[width=\textheight]{img/expH.pdf}
$n=5$

\end{frame}

%------------------------------------------------

\begin{frame}
\frametitle{Results obtained}

\begin{itemize}

\item The results obtained have a bigger error and use more size.
\item It seems that there is no advantage in using different precisions in 
individual elements.
\item The utility function can be used to compare the results

\end{itemize}

\end{frame}

%------------------------------------------------

\begin{frame}
\frametitle{Utility function}

Assuming that we are interested only in the reduction of \textbf{storage size} 
$s$, while maintaining a low error, the utility function $u$ can be defined as:
$$ u = \textrm{error} + \textrm{size} $$
However, both quantitites need to be scaled acordingly, so that a unit reduction 
of error, is equally good as a unit of reduction of size.

\end{frame}

%------------------------------------------------

\begin{frame}
\frametitle{Scaling error and size}

We can use the relation between error and space $s$ that we measured 
experimentally.
$$ \log_2 \Delta \approx -b + \alpha \log_2 n $$
Then, $u$ becomes
$$ u(\Delta, b, n) = \log_2 \Delta + b - \alpha \log_2 n $$
where $b$ is the mean bit-width, $n$ is the size of the input $n\times n$ 
matrix, and $\alpha = 2.78857$.

\end{frame}

%------------------------------------------------
%------------------------------------------------

\begin{frame}

\centering
\includegraphics[width=\textheight]{img/exp10-u.pdf}
$n=5$

\end{frame}
%------------------------------------------------

\begin{frame}

\centering
\includegraphics[width=\textheight]{img/exp10-bestu.pdf}
$n=5$

\end{frame}

%------------------------------------------------

\begin{frame}
\frametitle{Conlusions}

The best results (from the point of view of the utility function) are those 
that:
\begin{itemize}
\item Maintain the same precision in individual elements
\item Use all the variables to the same value $b$
\item Except for the scale variable, which has almost no influence in the error.
\end{itemize}

\end{frame}
%------------------------------------------------

\begin{frame}
\frametitle{Caveats and posible solutions}

The current utility function does not measure the time nor the energy. A posible 
solution could be done by simulation.

\begin{itemize}
\item A simulation can estimate the time of the floating point units with 
bit-width $b$.
\item The IO operations in RAM and cache can be simulated and measured.
\item The conversion between different bit-width variables needs to be also 
accounted.
\end{itemize}

\end{frame}
%------------------------------------------------
\end{document}
